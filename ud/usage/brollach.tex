\newpage

\chapter*{Díonbhrollach}

Is frásleabhar trítheangach é seo: Manainnis, Gaeilge, agus Béarla.
Is é Phil Kelly a rinne obair na gcapall chun na samplaí Béarla/Manainnis
a bhailiú ó fhoclóirí éagsúla, an chuid is mó acu ó 
\textit{Fargher's English-Manx Dictionary} a foilsíodh sa bhliain 1979.
Tá na sonraí Béarla/Manainnis ar fáil in áiteanna éagsúla ar an
Idirlíon cheana féin,
mar shampla ar an suíomh iontach \texttt{taggloo.im}.
Na rudaí a rinne mise anseo ná: (1) Gaeilge a chur ar na frásaí go léir, agus 
(2) iad a eagrú faoi cheannfhocail sa Manainnis.

Mar a tharlaíonn sé, bhain an tUas. Fargher an-úsáid as struchtúr an 
\textit{English-Irish Dictionary} a chuir Tomás de Bhaldraithe in eagar
sa bhliain 1959: an liosta ceannfhocal, struchtúr inmheánach na
n-iontrálacha, agus na samplaí úsáide féin.
Chomh maith leis sin,
is minic gur chum Fargher téarmaí agus frásaí nua nach raibh sa Manainnis
chlasaiceach, bunaithe go díreach ar na leaganacha Gaeilge
i bhFoclóir de Bhaldraithe. Is d'aon turas a rinne sé seo, agus dúirt
sé go soiléir sa bhrollach gur foclóir ``prescriptive'' seachas
``descriptive'' a bhí ann.
Is ionann é seo agus an straitéis a chuir de Bhaldraithe i bhfeidhm
ar go leor bealaí, ag baint úsáide as
\textit{Harrap's French-English Dictionary}
mar bhunús. Tá go leor téarmaí nua i bhfoclóir de Bhaldraithe
nach raibh sa teanga bheo roimhe sin, mar áis d'aistritheoirí,
múinteoirí, agus daoine eile a bhí ag plé leis an nGaeilge sa
saol nua-aimseartha ag an am.

Seo sampla amháin ó na trí fhoclóir a léiríonn cé chomh mór a bhí siad
ag brath ar a chéile:

\vspace{1.4ex}

\textbf{stare}\textsuperscript{1} [st\textepsilon\textschwa{}r], s. Regard fixe; regard appuye \textit{Glassy s.}, regard terne, vitreux. \textit{Set s.}, regard fixe. \textit{Stony s.}, regard dur. \textit{Vacant s.}, regard vague; regard ahuri. \textbf{To give s.o. a stare}, dévisager qn. \textit{With a s. of astonishment}, les yeux écarquillés, les yeux ébahis. \textit{With a s. of horror}, les yeux grands ouverts d'horreur.

\vspace{0.8ex}

\textbf{stare}\textsuperscript{1}, s. Stánadh \textit{m}, lán \textit{m} na súl. \textbf{Glassy stare}, \textit{fiarshúil ghlinniúnach}. \textbf{Stony stare}, \textit{amharc m fuar}. \textbf{Vacant stare}, \textit{stánadh folamh}. \textbf{To give s.o. a stare}, \textit{stánadh ar dhuine; lán na súl a bhaint as duine}. \textbf{With a stare of astonishment}, \textit{agus seasamh ina shúile le hiontas}. \textbf{With a stare of horror}, \textit{agus dhá shúil mhóra ina cheann le tréan uafáis}.

\vspace{0.8ex}

\textbf{stare}, \textit{n.} blakey \textit{m.pl.} blakaghyn, lane \textit{(m)} ny sooilley. \textit{Stony stare}, Blakey \textit{m.} feayr. \textit{Glassy stare}, Blakey \textit{(m)} glessagh \textit{Vacant stare}, Blakey follym. \textit{To give s.o. a stare}, Lane ny sooillyn y chur da peiagh ennagh. \textit{With a stare of astonishment}, As shassoo ny sooillyn lesh ard-yindys.

\vspace{1.4ex}

Is léir ón sliocht seo gur tháinig roinnt de na leaganacha
Manainnise go díreach ó de Bhaldraithe.
Dá bharr seo, nuair a rinne mé ``aistriúchán'' ar na
frásaí Manainnise, is minic nach raibh i gceist ach ``athaontú''
a dhéanamh leis an leagan i bhfoclóir de Bhaldraithe.  I ngach cás eile,
chuir mé an leagan Manainnise tríd an inneall aistriúcháin Intergaelic,
ansin rinne mé glanadh éadrom ar an aschur.
Dá bharr sin, cloíonn na haistriúcháin go dlúth le struchtúr na Manainnise,
fiú nuair a bhíonn cor cainte níos nádúrtha Gaeilge ar fáil.
Mar shampla, is é ``Ó cheann go bonn'' an t-aistriúchán a thugaim ar 
``Voish kione gys boyn'', cé go mbeadh ``Ó bhonn go baithis'' níos
dúchasaí i nGaeilge. I ndeireadh an lae, is áis é seo do
chainteoirí Gaeilge atá foghlaim na Manainnise agus an sprioc a bhí 
agam ná leaganacha cainte agus gramadach na Manainnise a shoiléiriú.
Nuair nach raibh an t-aistriúchán díreach go hiomlán soiléir, 
chuir mé leagan nádúrtha Gaeilge leis idir lúibíní. Mar shampla,
an t-aistriúchán ar \textit{Ynsee ayns dty chree eh} ná
\textit{Foghlaim i do chroí é (``Foghlaim de ghlanmheabhair é'')}.

Tá roinnt botún i bhfoclóir Fargher mar gheall ar mhíthuiscint nó míléamh 
ar an leagan Gaeilge (is dócha). Mar shampla, 
\textit{Hie eh dy creoi ayn} (\textit{``It cut him to the quick''});
go litriúil \textit{Chuaigh sé go crua ann}, cé gurb é 
\textit{Chuaigh sé go croí ann} an leagan i bhfoclóir de Bhaldraithe.
Nó, \textit{Sooillyn lhieent lesh jeirnyn} (\textit{``Eyes swimming with tears''}), go litriúil \textit{Súile líonta le deora}, bunaithe ar
\textit{Súile ina linnte deor}. Tugaim an leagan ``ceart'' ó de Bhaldraithe
idir lúibíní i gcásanna den sórt seo.

Tá mé tar éis cúpla teicneolaíocht teanga nua a fhorbairt don Mhanainnis
a laghdaigh an obair mhaslach a bhí ag teastáil chun an frásleabhar
seo a chruthú. Luaigh mé an t-inneall aistriúcháin Intergaelic cheana.
Níos luaithe i mbliana, thóg mé corpas parsáilte mar chuid den 
tionscadal idirnáisiúnta \textit{Universal Dependencies}.
Bhí mé in ann an corpas sin a úsáid chun clibeálaí (ranna cainte) 
agus parsálaí (comhréire) a thraenáil agus a chur i bhfeidhm ar
na frásaí. Is é seo an tslí a raibh mé in ann na samplaí a eagrú faoi na
ceannfhocail chearta.
Is rud beag é seo ar bhealach, ach is mór an cúnamh é do
chainteoirí Gaeilge nach bhfuil mórán taithí acu
ar chóras na n-athruithe tosaigh sa Manainnis.
Chomh maith leis sin, beidh an clibeálaí agus an parsálaí 
an-áisiúil d'aon tionscadal foclóireachta don Mhanainnis
a tharlóidh sa todhchaí.  

Tá mé buíoch de mo mhac, Kevin Scannell, Jr., a thug an-chúnamh dom
agus an corpas atá taobh thiar den leabhar seo á chur le chéile.
\vspace{1.6ex}

\noindent Kevin Scannell\\
3 Eanáir 2021\\
Saint Louis

\endinput
